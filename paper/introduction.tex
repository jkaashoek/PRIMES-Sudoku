\section{Introduction}

Students learn by practicing over lots of problems, but generating fresh problems that involve using the same set of concepts and are of a given difficulty level requires a lot of work for a teacher. Our main goal is to automatically generate programming problems that are of different complexity and that are parameterized by the set of concepts a student might want to learn. In this work, we present a system that solves a simpler task of automatically generating sudoku puzzles of different complexity levels, but we have found our technique and algorithms general enough to generate programming problems as well other domains such as algebra and trigonometry problems.

We present a generic iterative constraint-based algorithm to generate problems of different complexity levels. Most previous approaches for automatically generating puzzle problems have been specific to a given puzzle and are based on a set of heuristic rules. Our approach, on the other hand, lets one specify the puzzle definition and puzzle complexity in a declarative fashion using constraints and then uses efficient constraint-solving to incrementally solve constraints generated from different iterations. The algorithm first starts with a complete random problem that satisfies the constraints that it is well-formed and covers the desired set of concepts. It then starts removing elements from the complete problem in a user-defined probabilistic fashion such that after each removal some problem-specific properties (such as bounded number of solutions, uniform distribution of spaces etc.) are satisfied. We use the z3 SMT solver [6] and its theory of linear arithmetic for representing and solving the constraints.

We have successfully used our system to automatically generate more than 200,000 9X9 sudoku puzzles varying across a number of features: number of empty spaces, number of solutions, distribution of empty spaces, repetition of digits etc. The declarative nature of our system lets us easily parameterize the algorithm to also generate 16X16 and 25X25 sudoku problems. Since computing the hardness of a sudoku problem is still an open research problem, we resort to machine learning techniques to learn a function over the sudoku features from a set of labelled sudoku problems obtained from popular sudoku websites and newspapers. We then use this learnt function to characterize the sudoku problems generated by our system into different complexity levels. We are currently extending our system to support generation of python programming problems as well as other puzzles.
