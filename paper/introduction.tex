\section{Introduction}

Students learn by practicing many problems, but generating fresh problems that have specific characteristics, such as using a certain set of concepts or being of a given difficulty level, is a
tedious task for a teacher. Our goal is to automatically generate programming problems that are parameterized by the characteristics most beneficial for a student to learn. In this paper, we present a system that solves a simpler task of automatically generating sudoku and fillomino puzzles of different complexity levels. We find our technique and algorithms general enough to generate programming problems as well as problems in other domains, such as algebra and trigonometry.

We present a generic, iterative, and constraint-based algorithm for generating problems of different complexity levels. Most previous approaches for automatically generating puzzle problems have been specific to a given puzzle and are based on a set of heuristic rules. Our approach, on the other hand, lets one specify the puzzle definition and puzzle complexity in a declarative fashion, using constraints and then efficient constraint-solving to incrementally satisfy restraints generated from different iterations. The algorithm first generates a completely random problem that satisfies the constraints. It then removes elements from the complete problem in a user-defined probabilistic fashion. We use the z3 SMT solver [6] and its theory of linear arithmetic for representing and solving the constraints.

We have successfully used our system to automatically generate more than 200,000 9x9 sudoku puzzles and more than 10,000 fillomino puzzles. Our puzzles vary across a number of features: number of empty spaces, number of solutions, distribution of empty spaces, repetition of digits etc. The declarative nature of our system lets us easily parameterize the algorithm to also generate puzzles of different sizes, such as 16x16 and 25x25 sudoku puzzles. Since computing the difficulty level of a sudoku puzzle is still an open research problem, we resort to machine learning techniques to learn a function over the sudoku features from a set of labelled sudoku problems obtained from popular sudoku websites and newspapers. We then use this function to characterize the sudoku problems generated by our system into different complexity levels. We are currently extending our system to support generation of python programming problems, math problems, and other puzzles.
