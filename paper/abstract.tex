\begin{abstract}

Engaging students in practicing a wide range of problems facilitates their learning. However, generating fresh problems that have specific characteristics, such as using a certain set of concepts or being of a given difficulty level, is a tedious task for a teacher. In this paper, we present a system $\puzzlejar$,
which is based on an iterative constraint-based technique for
automatically generating problems. The $\puzzlejar$ system takes as
parameters the problem definition, the complexity function, and
domain-specific semantics-preserving transformations. We present an
instantiation of our technique with automated generation of Sudoku and
Fillomino puzzles, and we are currently extending our technique to
generate Python programming problems. Since defining complexities of
sudoku and fillomino puzzles is still an open research question, we
developed our own mechanism to define complexity, using machine
learning to generate a function for difficulty from puzzles with
already known difficulties. Using this technique, $\puzzlejar$ generated over
200,000 sudoku puzzles of different sizes (9x9, 16x16, 25x25) and over
10,000 fillomino puzzles of sizes ranging from 2x2 to 16x16.

\end{abstract}
