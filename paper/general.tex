\title{Automatically Generating Puzzles of Different Complexity}
\author{}
\date{\today}

\documentclass[12pt]{article}

\usepackage{amsfonts}
\usepackage{amsthm}

\begin{document}
\maketitle

\newtheorem{definition}{Definition}
\newcommand \puzzleboard {\mathcal{P}}
\newcommand \puzzleval {V}
\newcommand \puzzledomain {\mathcal{D}}
\newcommand \natnumbers {\mathbb{N}}
\newcommand \puzzledecl {D}
\newcommand \puzzlecomplexity {C}
\newcommand \puzzletransformset {\tilde{T}}
\newcommand \complexityrange {H}
\newcommand \puzzletransform {T}


\section{General Approach}

\begin{definition}
We define a puzzle board $\puzzleboard$ to be a two dimensional $n
\times m$ board. Let $\puzzleval$ denote a function that assigns
values to the squares on the puzzle board such that the value of the
square $(i,j)$ is defined by $\puzzleval(i,j)$, where $1 \leq i \leq
n$, $1 \leq j \leq m$, $V: \natnumbers \times \natnumbers \rightarrow
\puzzledomain$, and $\puzzledomain$ denotes the set of all possible
values the puzzle squares can take.
\end{definition}

The main components of our technique are:
\begin{itemize}
\item{A declarative definition of puzzle $\puzzledecl$}
\item{A Complexity Function : $\puzzlecomplexity$}
\item{A set of Transformation Functions : $\puzzletransformset$}
\end{itemize}

\begin{definition}
A declarative definition of puzzle $\puzzledecl$ defines constraints over the set of valid values $\puzzleval(i,j)$ that puzzle squares can take.
\end{definition}

\begin{definition}
The complexity function $\puzzlecomplexity : \puzzleboard \rightarrow
\complexityrange$ takes a puzzle board $puzzleboard$ as input and maps
it to a finite class of hardness levels denoted by $\complexityrange$.
\end{definition}

\begin{definition}
A transformation function $\puzzletransform : \puzzleboard \rightarrow \puzzleboard$ takes a puzzle board as input and transforms it another puzzle board such that the new puzzle board also satisfies the puzzle constraints. The set of all transformation functions are denoted by $\puzzletransformset$.
\end{definition}

\begin{verbatim}
The General Puzzle Creation Algorithm
P_I = GetRandomPuzzle(P_D)

While isRemoveValid(P_c,i,j):
  P_c = P_c - {i,j}
  For T \in T_P:
    R = R U T(P_c)

For P \in R:
  D[F_c(P)] = P
\end{verbatim}

The algorithm first uses an off-the-shelf constraint solver to solve
the puzzle constraints $P_D$ to get an initial random board
configuration. It then starts removing the value at a square (i,j)
until there are no more square values remaining to be removed. The
algorithm uses the isValidSquare function to check if certain puzzle
constraints hold after removing the square value (i,j). An example
isRemoveOK function is that the Number of solutions to the puzzle
after removing the square value is only $1$. Let $P_c$ denote the puzzle
obtained after removing a valid square. We then apply a set of
transformation functions $T_P$ to get a set of new puzzles.


\end{document}
