\subsection{Overview of $\puzzlejar$}

We first present an overview of the $\puzzlejar$ system to synthesize
puzzles of varying complexity levels. $\puzzlejar$ takes three
components as input: (i) a declarative definition $\puzzledecl$ of the
puzzle $\puzzleboard$, (ii) a complexity function $\puzzlecomplexity$,
and (iii) a set of transformation functions $\puzzletransformset$. In
this section, we formally define these components and present our
general synthesis algorithm to automatically generate puzzles of
varying complexity.

\subsubsection{Definitions}

\begin{definition}[Puzzle]
A two dimensional puzzle board of size $n \times m$ is defined using a
valuation function $\puzzleboard: \natnumbers \times \natnumbers
\rightarrow \puzzledomain$, which assigns values to the squares on the
puzzle board. The value of a square $(i,j)$ on the puzzle board is denoted by
$\puzzleboard(i,j)$, where $1 \leq i \leq n$, $1 \leq j \leq m$, and
$\puzzledomain$ denotes the set of possible values the puzzle
squares can take.
\end{definition}

\begin{definition}[Declarative Definition]
The declarative definition of puzzle $\puzzledecl$ defines constraints over the set of valid values $\puzzleboard(i,j)$ that the puzzle squares can take.
\end{definition}

\begin{definition}[Complexity]
The complexity function $\puzzlecomplexity : \puzzleboard \rightarrow
\complexityrange$ takes a puzzle board $\puzzleboard$ as input and maps
it to a finite class of hardness levels denoted by $\complexityrange$.
\end{definition}

\begin{definition}[Transformations]
A transformation function $\puzzletransform : \puzzleboard \rightarrow
\puzzleboard$ takes a puzzle board as input and transforms it to
another puzzle board such that the new puzzle board also satisfies the
puzzle constraints $\puzzledecl$. The set of all transformation
functions are denoted by $\puzzletransformset$.
\end{definition}

\begin{algorithm}[!htpb]
\caption{$\puzzlegen$($\puzzledecl$, $\puzzlecomplexity$, $\puzzletransformset$)}
\begin{algorithmic}[1]
\STATE $\randomPuzzle := \getrandom(\puzzledecl)$
\STATE $\currentpuzzle := \randomPuzzle$
\STATE $\squarestested := \emptyset$
\STATE{$\complexitydict := \{\}$}
\WHILE{$|\squarestested| \neq |\randomPuzzle|$}
\STATE $(i,j) = \choosesquare(\currentpuzzle)$
\STATE $\squarestested = \squarestested \cup (i,j)$
\IF{$\isremovevalid(\currentpuzzle,i,j)$}
\STATE {\[ \currentpuzzle'(k,l) = \left\{ 
  \begin{array}{l l}
    \currentpuzzle(k,l) & \quad \mbox{if $(k,l) \neq (i,j)$}\\
    \phi & \quad \mbox{if $(k,l) = (i,j)$}
  \end{array} \right.\]
}
\STATE{$\allpuzzles := \emptyset$}
\FOR{$\puzzletransform \in \puzzletransformset$}
\STATE{$\allpuzzles := \allpuzzles \cup \puzzletransform(\currentpuzzle')$}
\ENDFOR
\FOR{$\puzzleboard \in \allpuzzles$}
\STATE{$h := \puzzlecomplexity(\puzzleboard)$}
\STATE{$\complexitydict[h] := \complexitydict[h] \cup \puzzleboard$}
\ENDFOR
\STATE{$\currentpuzzle := \currentpuzzle'$}
\ENDIF
\ENDWHILE
\RETURN{$\complexitydict$}
\end{algorithmic}
\end{algorithm}

\subsubsection{Constraint-based Synthesis Algorithm}

The algorithm first uses the $\getrandom$ function to generate an initial
random puzzle board configuration $\randomPuzzle$ by solving the
puzzle constraints $\puzzledecl$ using an off-the-shelf constraint
solver. It then starts emptying squares on the board one at a time
until all the squares have been tested, i.e. the size of the set
$|\squarestested|$ becomes equal to the size of the puzzle board
$|\randomPuzzle|$. The algorithm chooses the squares to be emptied
using the $\choosesquare$ function, which takes the current puzzle
board configuration $\currentpuzzle$ as an input. The $\choosesquare$
function uses a user-defined strategy to select a square that can vary
from being completely random to a strategy that selects squares based
on the distribution of the values of the current puzzle board. After a
square is selected to be removed, the algorithm checks whether certain
puzzle constraints are met after removing the chosen square using the
$\isremovevalid$ function. A common $\isremovevalid$ function is to
check if the current puzzle $\currentpuzzle$ has a unique solution,
but $\puzzlejar$ allows for any general $\isremovevalid$ function.

If the $\isremovevalid$ function returns \t{True}, i.e. if we still
get a valid puzzle after removing the square $(i,j)$ from the puzzle
$\currentpuzzle$, the algorithm creates a new puzzle $\currentpuzzle'$
that has the same square values as the puzzle $\currentpuzzle$ except
the square $(i,j)$ whose value is now set to $\phi$ (denoting an empty
square). Often times, we can apply puzzle semanticss-preserving
transformations to the puzzle boards to get new puzzle board
configurations. The algorithm applies the set of transformations
$\puzzletransformset$ to the new puzzle board $\currentpuzzle'$ to
obtain a set of puzzle boards $\allpuzzles$. Finally, the algorithm
computes the complexity of each puzzle board $h$ using the complexity
function $\puzzlecomplexity$ and assigns it to appropriate complexity
level in the dictionary $\complexitydict$. This $\complexitydict$ is
the resulting dictionary that is returned by the $\puzzlegen$
algorithm.

In general, it is difficult to provide a puzzle complexity function
$\puzzlecomplexity$ that can assign a hardness level to a puzzle
board. Even for relatively simpler puzzles such as the Sudoku, the
complexity function is an open research
question~\cite{sudokuchaos}. In $\puzzlejar$, we try to approximate
this complexity function using machine learning techniques. We obtain
a set of labelled training data that consists of a set of puzzle
configurations each labelled with a hardness label $h$. Currently we
use the puzzle data available in books/web/newspapers, but we plan to
obtain such labelled training data from human subjects in future. For
a puzzle, we define a feature vector consisting of a set of features
that may be useful to capture its complexity. We then use Support
Vector Machines to learn a function $\puzzlecomplexity$ that can map
the feature vectors of the puzzles in the training set to their
corresponding hardness levels.

$\puzzlejar$ allows for any $\isremovevalid$ function such as a
function that checks whether the number of current solutions is less
than a constant $k$. A general strategy to perform this check is to
use an off-the-shelf constraint solver to first find a solution $S$ to
the puzzle, and then solve for another solution $S'$ by adding an
extra constraint that the new solution should not be the previous
solution $S \neq S'$. For a value $k$, we get the constraint $S' \neq
S_1 \land S' \neq S_2 \land \cdots \land S' \neq S_k$. This strategy
needs $k+1$ (potentially exponential) solver calls to check whether a
square can be emptied from the puzzle board, which can make the
overall algorithm quite expensive. For the common case of $k=1$ (the
constraint that the puzzle should always have a unique solution), we
can perform this check efficiently using just a single solver call by
adding a constraint that $S' \neq \randomPuzzle$, i.e. check whether
there exists a solution $S'$ that is different from the original
puzzle board.

