\section{Overview of our Framework}

We present an overview of our general framework to synthesize puzzles
of varying complexity levels. Our framework takes three components as
input: a declarative definition of the puzzle $\puzzledecl$, a
complexity function $\puzzlecomplexity$, and a set of transformation
functions $\puzzletransformset$. We first define these components and
then present our synthesis algorithm to automatically generate
different puzzles.

\begin{definition}
A two dimensional puzzle board of size $n \times m$ is defined using a
valuation function $\puzzleboard: \natnumbers \times \natnumbers
\rightarrow \puzzledomain$ that assigns values to the squares on the
puzzle board. The value of the square $(i,j)$ is denoted by
$\puzzleboard(i,j)$, where $1 \leq i \leq n$, $1 \leq j \leq m$, and
$\puzzledomain$ denotes the set of all possible values the puzzle
squares can take.
\end{definition}

\begin{definition}
The declarative definition of puzzle $\puzzledecl$ defines constraints over the set of valid values $\puzzleboard(i,j)$ that the puzzle squares can take.
\end{definition}

\begin{definition}
The complexity function $\puzzlecomplexity : \puzzleboard \rightarrow
\complexityrange$ takes a puzzle board $\puzzleboard$ as input and maps
it to a finite class of hardness levels denoted by $\complexityrange$.
\end{definition}

\begin{definition}
A transformation function $\puzzletransform : \puzzleboard \rightarrow
\puzzleboard$ takes a puzzle board as input and transforms it to
another puzzle board such that the new puzzle board also satisfies the
puzzle constraints $\puzzledecl$. The set of all transformation
functions are denoted by $\puzzletransformset$.
\end{definition}

\begin{algorithm}[!htpb]
\caption{$\puzzlegen$($\puzzledecl$, $\puzzlecomplexity$, $\puzzletransformset$)}
\begin{algorithmic}[1]
\STATE $\randomPuzzle := \getrandom(\puzzledecl)$
\STATE $\currentpuzzle := \randomPuzzle$
\STATE $\squarestested := \Phi$
\WHILE{$|\squarestested| \neq |\randomPuzzle|$}
\STATE $(i,j) = \choosesquare(\currentpuzzle)$
\STATE $\squarestested = \squarestested \cup (i,j)$
\IF{$\isremovevalid(\currentpuzzle,i,j)$}
\STATE {\[ \currentpuzzle(k,l) = \left\{ 
  \begin{array}{l l}
    \currentpuzzle(k,l) & \quad \mbox{if $(k,l) \neq (i,j)$}\\
    \phi & \quad \mbox{if $(k,l) = (i,j)$}
  \end{array} \right.\]
}
\FOR{$\puzzletransform \in \puzzletransformset$}
\STATE{$\allpuzzles := \allpuzzles \cup \puzzletransform(\currentpuzzle)$}
\ENDFOR
\STATE{$\complexitydict := \{\}$}
\FOR{$\puzzleboard \in \allpuzzles$}
\STATE{$h := \puzzlecomplexity(\puzzleboard)$}
\STATE{$\complexitydict[h] := \complexitydict[h] \cup \puzzleboard$}
\ENDFOR
\ENDIF
\ENDWHILE
\RETURN{$\complexitydict$}
\end{algorithmic}
\caption{The $\puzzlegen$ algorithm for synthesizing puzzles of varying complexity levles.}
\end{algorithm}

The algorithm first uses the $\getrandom$ function to get an initial
random puzzle board configuration $\randomPuzzle$ by solving the
puzzle constraints $\puzzledecl$ using an off-the-shelf constraint
solver. Note that sometimes for this first step, we also use custom
random puzzle generators for scalability. It then starts removing the
value at a square (i,j) until there are no more square values
remaining to be removed. The algorithm uses the isValidSquare function
to check if certain puzzle constraints hold after removing the square
value (i,j). An example isRemoveOK function is that the Number of
solutions to the puzzle after removing the square value is only
$1$. Let $P_c$ denote the puzzle obtained after removing a valid
square. We then apply a set of transformation functions $T_P$ to get a
set of new puzzles.



