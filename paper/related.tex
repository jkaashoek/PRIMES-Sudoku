\section{Related Work}

With the advent of recent online education initiatives, we are seeing an ever increasing number of students enrolling in online classrooms. Some of the popular courses such as Introduction to Programming and Introduction to Machine Learning routinely reports more than 100,000 student enrollment. This large scale of students has forced us to develop new automated technologies to solve problems such as automated feedback generation [4] and solution generation[5]. Another important problem that comes up because of this scale is that of automated problem generation to cater to the practice needs of different students as well as for providing different exams to students of a given difficulty level.

There has been some previous work on generating new problems in various domains namely algebra and programming. The work on generating new algebra problem has mostly been looked upon using two main approaches. In the first approach, a teacher is provided a certain set of parameter values that are fixed for a given domain [2]. For example, for generating a quadratic equation, the parameters can be the number of roots, difficulty of factorization, whether there is an imaginary root, the range of coefficient values etc. Given a set of feature valuations, the tool generates the corresponding quadratic equations. The second approach takes a particular proof problem, and tries to learn a problem template from the problem which is then instantiated with different concrete values [3]. The system first tries to learn a general query from a given proof problem, which is then executed to generate a set of proof problems. Since the query is only a syntactic generalization of the original problem, only a subset of them are valid problems, which are identified using polynomial identity testing. Our approach, however, creates different versions of the same problem by introducing different number of holes in the original problem based on a parametric complexity function.

More recently, a technique was proposed to generate fill-in-the-blank Java problems where certain keywords, variables and control symbols are removed randomly from a correct solution [1]. The technique blanks variables using the condition that at least one occurrence of each variable remains in the scope and blanks control symbols such that at least one occurrence of a paired symbol (such as brackets) remains. Our technique, on the other hand, is more general since it is constrained-based and can check for more interesting constraints such as unique solutions and an arbitrary complexity function that it takes as an additional parameter.
