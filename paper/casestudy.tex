\section{Case Studies}

\subsection{Sudoku}

\subsubsection{Declarative Definition}
We use the python frontend of the z3 constraint solver to specify the 9x9
sudoku puzzle declaratively. As can be noticed from the encoding, it
can be easily generalized to other sudoku sizes, such as 16x16 or
25x25.

We first define 81 different integer variables (\t{X[0][0], X[0][1], \ldots,
X[8][8]}), where \t{X[i][j]} denotes the value of the sudoku cell (i, j). We
also define the valid set of values each element can take, i.e. 1 $\leq$
\t{X[i][j]} $\leq$ 9 (valid values).

\singlespace
\begin{lstlisting}[language=python, frame = single]
X = [[Int('x%d%d' % (i,j)) for i in range(9)] for j in range(9)]
valid_values = [And ( X[i][j] >= 1, X[i][j] <= 9) for i in range(9)
for j in range(9)]
\end{lstlisting}
\doublespace

We now add the sudoku constraints that the values in each row should be
distinct (\t{rows\textunderscore distinct}), values in each column should be distinct
(\t{cols\textunderscore distinct}), and that values each 3x3 square should be distinct
(\t{three\textunderscore by\textunderscore three\textunderscore distinct}).

\singlespace
\begin{lstlisting}[language=python, frame = single]
row_distinct = [Distinct(X[i]) for i in range(9)]
cols_distinct = [Distinct([X[i][j] for i in range(9)]) for j in
range(9)] 
three_by_three_distinct = [ Distinct([X[3*k + i][3*l + j] for i in
range(3) for j in range(3)]) for k in range(3) for l in range(3)]

\end{lstlisting}
\doublespace

To encode partially filled sudoku board (where a 0 value denotes an
empty space), we simply add the constraint \t{X[i][j] == board[i][j]} when
\t{board[i][j]} != 0. 

\singlespace
\begin{lstlisting}[language=python, frame = single]
 already_set = [X[i][j] == board[i][j] if board[i][j] != 0 for i in
 range(9) for j in range(9)]

\end{lstlisting}
\doublespace

The complete set of constraints sudoku constraint is obtained by
combining all previous constraints:

\singlespace
\begin{lstlisting}[language=python, frame = single]
sudoku_constraint = valid_values + row_distinct + cols_distinct +
three_by_three_distinct + already_set
\end{lstlisting}
\doublespace


\subsubsection{Creating the Initial Puzzle}
To accomplish this, we use a set of equations using the python z3 constraint solver. This would generate a sudoku board in a 2 dimensional list such as the one
below. 
\t{
\singlespace
\begin{center}
[[4, 9, 7, 1, 8, 2, 5, 3, 6] \newline
[1, 5, 2, 3, 6, 4, 8, 9, 7] \newline
[8, 6, 3, 5, 7, 9, 4, 1, 2] \newline
[7, 3, 4, 6, 9, 1, 2, 5, 8] \newline
[2, 8, 9, 4, 3, 5, 7, 6, 1] \newline
[5, 1, 6, 7, 2, 8, 9, 4, 3] \newline
[3, 2, 5, 9, 1, 7, 6, 8, 4] \newline
[9, 7, 1, 8, 4, 6, 3, 2, 5] \newline
[6, 4, 8, 2, 5, 3, 1, 7, 9]] \newline
\end{center}}
\doublespace

\subsubsection{Emptying Squares}

The next step is to start emptying this board. Our method for
selecting the next square to remove includes a number of
sub-steps. First, we calculate the percentage of squares in each row that are filled in ((number of squares in row that are full)/9). We then randomly
select one of these percentages and generate a random number between 0
and 1. If this number is greater than the percentage, we find a new
percentage and a new decimal between 0 and 1. Once we have a number
between 0 and 1 that is less than the percentage calculated we keep
this row. We then follow the same process to find which column in the
row we should empty. We set the square in the selected row and column
equal to zero, signifying emptiness.


After we have found a square to empty, we generate a temporary board
with the selected square  having a value of 0. We also a new set of
constraints to go along with this board using z3. If z3 can solve the
new set of equations in less than K ways, we have a desired result and
we keep this board. If z3 can generate a number of solutions that is
greater than or equal to K, the chosen square is added to a list of
squares that do not work and should not be tried again. This temporary
board is discarded and we find another square using the board we had
before the temporary board was generated. Once we find a square that,
when removed, results in a desired board, we find another square to
empty. If no other squares work, meaning that the number of squares
emptied plus the number of squares in the list of squares that do not
work equals 81, we stop looking for another square to empty. If the
algorithm does not stop in the first case, it stops once we reach the
number of squares we wish to empty, although this second case is less
likely if the desired number of empty squares is greater than 60.

\subsubsection{Defining Complexity}
After a Sudoku puzzle is generated, we determine its difficulty using
machine learning. For each puzzle, a vector is generated with
components describing the unsolved board. These characteristics are:

\singlespace
\begin{itemize}
\item Number of solutions
\item Number of empty squares
\item Number of rows with at least seven blank squares
\item Number of columns with at least seven blank squares
\item Number of 3x3 grids with at least seven blank squares
\item Number of occurrences of each digit (9 components)
\item Standard deviation of number of occurrences of each digit from the mean number of occurrences
\end{itemize}
\doublespace

The SVM library by scikit-learn then uses the vector to categorize the
puzzle into one of four difficulties: (1) Easy, (2) Medium, (3) Hard,
and (4) Evil.

\subsubsection{Transformations}
We are able to quickly generate more full Sudoku boards for emptying
without using SAT solvers. To do this, we apply the following
mathematically symmetrical transformations on an already-created
Sudoku board:

\singlespace
\begin{enumerate}
\item Relabeling the nine digits
\item Permuting the three 3x9 stacks
\item Permuting the three 9x3 bands
\item Permuting the three rows within a stack
\item Permuting the three columns within a band
\item Reflecting about the axes of symmetry in a square
\item Rotation by 90 degrees
\end{enumerate}
\doublespace
 
We chose a random transformation and applied it to an already existing
board 1000 times to create a new Sudoku board satisfying the original
constraints. Out of about 6 x 1021 total unique 9x9 Sudoku boards,
these transformations can generate about 3 x 106 new unique Sudoku
boards from an existing one.

We can apply most or all of these transformations to already-created
12x12, 15x15, 16x16, and 25x25 boards. On a 12x12 board, for instance,
the following transformations are analogous:

\singlespace
\begin{enumerate}
\item Relabeling the twelve digits
\item Permuting the four 3x12 stacks
\item Permuting the three 12x4 bands
\item Permuting the three rows within a stack
\item Permuting the four columns within a band
\item Reflecting about the horizontal and vertical axes in a square
\end{enumerate}
\doublespace
 
This method of quickly generating new boards generalizes to any
higher-numbered Sudoku board. 


The last step of our algorithm is quickly generating more sudoku
puzzles from the board that was emptied. This task is accomplished by
performing transformation on the existing board as described above.


\subsection{Fillomino}

\subsubsection{Declarative Definition}
Using the python frontend on z3, we can declaratively create a Fillomino puzzle.
First, we define an NxN board and assert that only the values between
1 and N can be on the board: 

\singlespace
\begin{lstlisting}[language=python]
cells = [[Int("x%d%d" % (i,j)) for i in range(1,N+1)] for j in
                range(1,N+1)]

valid_cells = [And(cells[i][j] <= N, cells[i][j] >=1) for i in
range(N) for j in range(N)]
\end{lstlisting}
\doublespace

Now we must assert that, for each region on the board, the value of
all of the squares in a specified region must be the same as the
number of squares in that region.
 

To do this, we first create a variable edgeval that can either be one,
representing that an outgoing edge to an adjacent square exists or 0,
showing that the edge does not exist. An outgoing side should only
exist between two cells that are in the same region. The variable for
an edge is edgevar(i,j,k,l), where (i,j) is the location of the
original square and (k,l) is the location of the adjacent square.

\singlespace
\begin{lstlisting}[language=python]
for i in range(N):
    for j in range(N):
        for (k,l) in getAdjacent1(i,j):
            edge_var[(i,j,k,l)] = Int("e%d%d%d%d" % (i,j,k,l))
edge_val_constraints = [Or(edge_val==0,edge_val==1) for edge_val in
edge_var.values()]
\end{lstlisting}
\doublespace

For our method to work, we only want one outgoing edge from every
square, meaning that the value of edgevar will always be <= 0.

\singlespace
\begin{lstlisting}[language=python]
for i in range(N):
   for j in range(N):
      for (k,l) in getAdjacent1(i,j, N):
         if lessThan(i,j,k,l):
            sum_edges = edge_var[(i,j,k,l)] + edge_var[(k,l,i,j)]
                    edge_val_constraints.append(sum_edges <= 1)
\end{lstlisting}
\doublespace

Now we create a new variable for each cell, incell, that is equal to
the number of edges going into the cell.

\singlespace
\begin{lstlisting}[language=python]
in_cell = [[Int("in%d%d"%(i,j)) for i in range(N)] for j in range(N)]

    in_cell_constraints=[]
    for i in range(N):
        for j in range(N):
            in_cell_constraints.append(And(in_cell[i][j]>=0,
            in_cell[i][j]<=1))
            sum_incoming_edges = Sum([edge_var[(k,l,i,j)] for (k,l) in
            getAdjacent1(i,j, N)])
            in_cell_constraints.append(in_cell[i][j]==sum_incoming_edges)
\end{lstlisting}
\doublespace

Next we say that if there is an edge between two cells, those two
cells are in the same regions, and therefore, they should have the
same value.

\singlespace
\begin{lstlisting}[language=python]
  same_value_constraint = []
    for i in range(N):
        for j in range(N):
            for (k,l) in getAdjacent1(i,j, N):
                if lessThan(i,j,k,l):
                    same_value_constraint.append(Implies(Or(edge_var[(i,j,k,l)]==1,
                    edge_var[(k,l,i,j)]==1), cells[i][j] ==
                    cells[k][l]))

\end{lstlisting}
\doublespace

Finally, we define a cell’s size as being the sum of the sizes of the
adjacent cells + 1. We want there to be exactly one cell whose size is
the same as its value.

\singlespace
\begin{lstlisting}[language=python]
size_region_constraint=[]
size_cell = [[ Int("size%d%d" % (i,j)) for i in range(N)] for j in
                            range(N)]
    size_region_constraint=[And(size_cell[i][j] >=1,
    size_cell[i][j]<=N) for i in range(N) for j in range(N)]


    for i in range(N):
        for j in range(N):
            size_adjacent = Sum([If(edge_var[(i,j,k,l)]==1,
            size_cell[k][l],0) for (k,l) in getAdjacent1(i,j, N)])
            size_region_constraint.append(size_cell[i][j] ==
            (size_adjacent+1))
            size_region_constraint.append(Implies(in_cell[i][j]==0,
            size_cell[i][j] == cells[i][j]))
\end{lstlisting}
\doublespace

\subsubsection{Creating the Initial Puzzle}
We have two option to create the initial puzzle. We can run the
declarative definition of a Fillomino puzzle, but this code is slow
and will only get as puzzle because z3 does not come up with a
different solution each time the code is run, it is the same one every
time. Our second option is to randomly create a completed Fillomino
puzzle using python without any z3. The python code works by selecting
a starting square, randomly selecting a sequence length from a list of
valid sequence lengths, and then setting a list of squares that are
sequence length long that all have the value sequence length. We
continue like this until the board is completely filled.



\subsubsection{Choosing a Valid Square}
Choosing a Valid Square
Our first step is define what we mean by a valid square. For most
cases, we want to find a square that, when removed, results in a board
that has exactly one solution, but in general, we want it to have less
than K solutions. Our first step is to randomly choose a square on the
board, with the only restriction being that every region must have at
least one cell that is not emptied, as in the rules of Fillomino. 

In Sudoku, a square in a row, column, or three by three grid that had
less emptied squares in it already was more likely to be chosen. This
is not the case in Fillomino because the number of squares in a row or
column do not matter, the only constraint is that there must be at
least one value in every region that is not emptied.
 
As in Sudoku, we create a temporary board that has the selected square
removed (the square has a value of 0). We then apply the z3
declarative definition on this temporary board, and if z3 can find K
or more ways to solve the puzzle, we know that we should not remove
the square. We resort back to the old board, add the square that we
tried to a list of squares that we know do not work, and try again
with a new random square. This process continues until we have a
desired number of squares emptied.

\subsubsection{Defining Complexity}
Using machine learning, we determine the difficulty of an unsolved
fillomino puzzle based on the following characteristics:

\singlespace
\begin{itemize}
\item Number of cells
\item Number of empty squares
\item Number of regions (connected areas of the same value)
\item Optimality (whether the puzzle can be emptied further)
\end{itemize}

\doublespace
\subsubsection{Transformations}
Because of the randomness of the different regions, there are less
transformations that can be applied to Fillomino than Sudoku, but a
number still exist. These are:

\singlespace
\begin{enumerate}
\item Rotation
\item Vertical reflection
\item Horizontal reflection
\end{enumerate}
\doublespace

Applying these transformation to the original z3 declarative
definition puzzles as well as the randomly generating python generated
puzzles result in a fast way to create many more boards. The
transformation can also be applied to an emptied board to get more
solutions.

