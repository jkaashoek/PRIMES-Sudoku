\section{CaseStudies}

\subsection{Sudoku}

After a Sudoku puzzle is generated, we determine its difficulty using machine learning. For each puzzle, a vector is generated with components describing the unsolved board. These characteristics are:
\begin{itemize}
\item Number of solutions
\item Number of empty squares
\item Number of rows with at least seven blank squares
\item Number of columns with at least seven blank squares
\item Number of 3x3 grids with at least seven blank squares
\item Number of occurrences of each digit (9 components)
\item Standard deviation of number of occurrences of each digit from the mean number of occurrences
\end{itemize}

The SVM library by scikit-learn then uses the vector to categorize the puzzle into one of four difficulties: (1) Easy, (2) Medium, (3) Hard, and (4) Evil.

\subsection{Fillomino}

Using machine learning, we determine the difficulty of an unsolved fillomino puzzle based on the following characteristics:

\begin{itemize}
\item Number of cells
\item Number of empty squares
\item Number of regions (connected areas of the same value)
\item Optimality (whether the puzzle can be emptied further)
\end{itemize}