\section{Conclusion}

We present the $\puzzlejar$ system, based on a constraint-based
iterative algorithm, to automatically generate new fill-in-the-blank
type problems. We used Sudoku and Fillomino puzzles as case studies,
as their descriptions fit naturally with constraint solvers. The
$\puzzlejar$ system was able to generate hundreds of thousands of both
types of puzzles over different sizes. We are currently extending
$\puzzlejar$ to also create Python programming problems, on which we
already have some initial results. We believe a constraint-based
approach provides a generic and flexible mechanism for teachers to
specify different constraints that they would like a problem to have;
we can then use efficient constraint solvers to automatically generate
new problems.
