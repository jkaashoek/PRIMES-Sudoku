\documentclass[12pt]{article}

\usepackage{amsfonts}
\usepackage{amsthm}
\usepackage{algorithm}
\usepackage{algorithmic}
\usepackage{graphicx}
\usepackage{listings}
\usepackage{setspace}
\usepackage[margin=1in]{geometry}
\usepackage{color}
\usepackage{tabularx}
\definecolor{bluekeywords}{rgb}{0.13,0.13,1}
\definecolor{greencomments}{rgb}{0,0.5,0}
\definecolor{redstrings}{rgb}{0.9,0,0}

\date{}

\begin{document}
\doublespacing
\newtheorem{definition}{Definition}
\newcommand \puzzleboard {\mathcal{P}}
\newcommand \puzzleval {V}
\newcommand \puzzledomain {\mathcal{D}}
\newcommand \natnumbers {\mathbb{N}}
\newcommand \puzzledecl {D}
\newcommand \puzzlecomplexity {C}
\newcommand \puzzletransformset {\tilde{T}}
\newcommand \complexityrange {H}
\newcommand \puzzletransform {T}
\newcommand \randomPuzzle {\puzzleboard_I}
\renewcommand \t[1] {\texttt{#1}}
\newcommand \getrandom {\t{GetRandomPuzzle}}
\newcommand \currentpuzzle {\puzzleboard_C}
\newcommand \squarestested {\t{SquaresTested}}
\newcommand \choosesquare {\t{ChooseSquare}}
\newcommand \isremovevalid{\t{isRemoveValid}}
\newcommand \allpuzzles {R}
\newcommand \complexitydict {\t{complDict}}
\newcommand \puzzlegen {\t{GenPuzzles}}


\title{$\puzzlejar$: Automated Constraint-based Generation of Puzzles of Varying Complexity}

\maketitle

Engaging students in practicing a wide range of problems facilitates their learning. However, generating fresh problems that have specific characteristics, such as using a certain set of concepts or being of a given difficulty level, is a repetitious task for a teacher. Our goal is to develop a system to automatically generate problems in a large number of domains, including programming and math. 

In this work, we present $\puzzlejar$, a system that solves a simpler task of automatically generating
Sudoku and Fillomino puzzles of different complexity levels. Most previous
approaches for automatically generating puzzle problems have been
specific to a given puzzle and are based on a set of heuristic
rules. $\puzzlejar$, on the other hand, lets one specify the puzzle
definition and puzzle complexity using constraints. It then uses efficient constraint-solving to
incrementally solve constraints generated from different
iterations. The system first generates a solved, complete puzzle. It then introduces ``holes" into the puzzle to make it an interesting problem in a fashion such that the certain validity constraints are still satisfied. We use the z3 SMT solver  and its theory of linear arithmetic for representing
and solving the constraints.

We have successfully used $\puzzlejar$ to automatically generate more
than 200,000 9x9 Sudoku puzzles and more than 10,000 Fillomino
puzzles. The case studies with Sudoku and Fillomino puzzles show the generality of the algorithm, and our experiments confirm that $\puzzlejar$ can be reasonably applied to generate larger-sized puzzles.

\end{document}