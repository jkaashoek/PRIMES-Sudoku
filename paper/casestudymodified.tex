\subsection{Case Studies}

We now instantiate our general synthesis technique in $\puzzlejar$ on
two puzzles: Sudoku and Fillomino. For each puzzle, we present the
three components: its declarative definition, the features for
defining the complexity function, and the set of semantics-preserving
transformations.


\subsubsection{Sudoku}

An example Sudoku problem generated by $\puzzlejar$ together with its
solution is shown in Figure.~\ref{Sudokuprobsol}. The 9x9 Sudoku puzzle constraints are that each row, each column, and each 3x3 square should on the puzzle board should take distinct values from $1$ to $9$. A more formal definition for the Sudoku puzzle can be found in~\cite{Sudokudef}.

\subsubsection*{Declarative Definition}
We use the python frontend of the z3 constraint solver in combination
with list comprehension to specify the 9x9 Sudoku puzzle
declaratively. As can be noticed from the encoding, it can be easily
generalized to other Sudoku sizes, such as 16x16 or 25x25.

We first define 81 different integer variables (\t{X[0][0], X[0][1], \ldots,
X[8][8]}), where \t{X[i][j]} denotes the value of the Sudoku cell (i, j). We
also define the valid set of values each element can take: 1 $\leq$
\t{X[i][j]} $\leq$ 9 (valid values).

\singlespace
\begin{lstlisting}[language=python, frame = single]
X = [[Int('x%d%d' % (i,j)) for i in range(9)] for j in range(9)]
valid_values = [And ( X[i][j] >= 1, X[i][j] <= 9) for i in range(9)
for j in range(9)]
\end{lstlisting}
\doublespace

We now add the Sudoku constraints that the values in each row should be
distinct (\t{rows\textunderscore distinct}), values in each column should be distinct
(\t{cols\textunderscore distinct}), and that values each 3x3 square should be distinct
(\t{three\textunderscore by\textunderscore three\textunderscore distinct}).

\singlespace
\begin{lstlisting}[language=python, frame = single]
row_distinct = [Distinct(X[i]) for i in range(9)]
cols_distinct = [Distinct([X[i][j] for i in range(9)]) for j in
range(9)] 
three_by_three_distinct = [ Distinct([X[3*k + i][3*l + j] for i in
range(3) for j in range(3)]) for k in range(3) for l in range(3)]

\end{lstlisting}
\doublespace

To encode partially filled Sudoku board (where a 0 value denotes an
empty space), we simply add the constraint \t{X[i][j] == board[i][j]} when
\t{board[i][j]} != 0. 

\singlespace
\begin{lstlisting}[language=python, frame = single]
 already_set = [X[i][j] == board[i][j] if board[i][j] != 0 for i in
 range(9) for j in range(9)]

\end{lstlisting}
\doublespace

The complete set of Sudoku constraints is obtained by
combining all previous constraints:

\singlespace
\begin{lstlisting}[language=python, frame = single]
Sudoku_constraint = valid_values + row_distinct + cols_distinct +
three_by_three_distinct + already_set
\end{lstlisting}
\doublespace


\subsubsection*{Creating the Initial Puzzle}
We use the z3 constraint solver to generate a Sudoku board in a 2
dimensional list such as the example shown in Figure.~\ref{Sudokuprobsol}(b)%% below.  \t{
%% \singlespace
%% \begin{center}
%% [[4, 9, 7, 1, 8, 2, 5, 3, 6], \newline
%% [1, 5, 2, 3, 6, 4, 8, 9, 7], \newline
%% [8, 6, 3, 5, 7, 9, 4, 1, 2], \newline
%% [7, 3, 4, 6, 9, 1, 2, 5, 8], \newline
%% [2, 8, 9, 4, 3, 5, 7, 6, 1], \newline
%% [5, 1, 6, 7, 2, 8, 9, 4, 3], \newline
%% [3, 2, 5, 9, 1, 7, 6, 8, 4], \newline
%% [9, 7, 1, 8, 4, 6, 3, 2, 5], \newline
%% [6, 4, 8, 2, 5, 3, 1, 7, 9]] \newline
%% \end{center}}
%% \doublespace

\subsubsection*{Emptying Squares}

The next step in our algorithm is to start removing values from the
board. For Sudoku, our method for
selecting the next square to remove includes the following
sub-steps:

\begin{enumerate}
\item {\bf Select a square to empty in a probabilistic fashion.} For row $N$, we calculate the percentage of cells that have not been removed, $P_N$. We then randomly select a row $i$, where $0 \le i \le 8$, and generate a random number between 0 and 1. If this number is less than $P_i$, then we keep $i$ as our selected row. If this number is greater than $P_i$, then we discard $i$ and select a new random row and a random number. Once we have a number that is less than the percentage calculated we keep this row. We then follow the same process to find which column $j$ in the row we should empty. We choose $(i, j)$ as the square we wish to empty. The complete process allows for rows with more un-emptied cells to have a greater change of being chosen.
\item {\bf Temporarily set the selected square as emptied.} We create a temporary board and a new set of z3 constraints with the changed value.
\item{\bf Check whether the temporary board is valid.} If the temporary board has fewer than $k$ solutions, we keep the board. Otherwise, the selected square is added to \t{vals\_tried}, the list of squares that do not work, and we repeat steps 1 - 3 until we have a valid selected square.
\item{\bf Permanently remove a valid square.}
\item{\bf Repeat steps 1 through 4} until the target number of emptied squares is reached or the sum of the number of squares in \t{vals\_tried} and the number that are already emptied reaches 81. 
\end{enumerate}

\subsubsection*{Transformations}


\singlespace
\begin{table}
\begin{tabularx}{\textwidth}{|X|X|}
\hline
\multicolumn{2}{|c|}{Table 1: Symmetrical Transformations}\\
\hline
9x9 Sudoku Board & 12x12 Sudoku Board
\\
\hline
\begin{enumerate}
\item Relabeling the nine digits
\item Permuting the three 3x9 stacks
\item Permuting the three 9x3 bands
\item Permuting the three rows within a stack
\item Permuting the three columns within a band
\item Reflecting about the axes of symmetry in a square
\item Rotation by 90 degrees
\end{enumerate} &

\begin{enumerate}
\item Relabeling the twelve digits
\item Permuting the four 3x12 stacks
\item Permuting the three 12x4 bands
\item Permuting the three rows within a stack
\item Permuting the four columns within a band
\item Reflecting about the horizontal and vertical axes in a square
\end{enumerate}
 \\
\hline
\end{tabularx}
\end{table}

\doublespace

We are able to quickly generate more full Sudoku boards for emptying
without using the SMT solver. To do this, we repeatedly apply
mathematically symmetrical transformations to an already-created
Sudoku board. Out of about $6 \times 10^{21}$ total unique 9x9 Sudoku
boards, these transformations can generate about $3\times 10^6$ new
unique Sudoku boards from an existing one~\cite{sudokumath}. Furthermore, we
can apply most or all of these transformations to larger boards. Table
1 shows the symmetrical transformations that can be applied to 9x9 and
12x12 boards.

As the last step of our generation algorithm, we perform these transformations again on an already emptied board to generate more emptied puzzles that possibly have a different number of solutions.


\subsubsection*{Defining Complexity}
After a Sudoku puzzle is generated, we determine its difficulty using
machine learning. For each puzzle we use, a 15-component vector is generated to describe the unsolved board. These components are (1) number of solutions; (2) number of empty squares; (3) number of rows with at least seven blank squares; (4) number of columns with at least seven blank squares; (5) number of 3x3 grids with at least seven blank squares; (6 - 14) number of occurrences of each digit; (15) standard deviation of number of occurrences of each digit from the mean number of occurrences.

The SVM library by scikit-learn [citation] then uses the vector to categorize the
puzzle into one of four difficulties: (1) Easy, (2) Medium, (3) Hard,
and (4) Evil.

\subsubsection{Fillomino}

An example Fillomino problem generated by $\puzzlejar$ together with its
solution is shown in Figure.~\ref{Fillominoprobsol}. The Fillomino puzzle constraints are that the puzzle board should be divided into regions such that the square value of the cells in a region should all have the same value, where the value is equal to the size of the region. A more formal definition for the Sudoku puzzle can be found in~\cite{fillominodef}.

\subsubsection*{Declarative Definition}
Using the z3 Python frontend, we declaratively define a Fillomino puzzle.
First, we define an NxN board and assert that only the values between
1 and N can be on the board: 

\singlespace
\begin{lstlisting}[language=python, frame=single]
cells = [[Int("x%d%d" % (i,j)) for i in range(1,N+1)] for j in
         range(1,N+1)]
valid_cells = [And(cells[i][j] <= N, cells[i][j] >=1) for i in
	range(N) for j in range(N)]
\end{lstlisting}
\doublespace

Now we must assert the definition of a Fillomino puzzle, that the
value of all of the squares in a specified region must be the same as
the number of squares in that region. We do this using the concept of
spanning trees in the graphs, with each vertex being a cell on the board
and each edge representing some relationship between the cells.

We define each \t{edge\_val} to be a directed edge between a cell and
one of its adjacent cells. If \t{edge\_val == 1}, there exists an
outgoing edge from that cell to an adjacent square, and if
\t{edge\_val == 0}, there exists no such edge. We constrain it so that
outgoing edges exist only between two cells that are in the same
region.

\singlespace
\begin{lstlisting}[language=python, frame=single]
for i in range(N):
    for j in range(N):
        for (k,l) in getAdjacent1(i,j):
            edge_var[(i,j,k,l)] = Int("e%d%d%d%d" % (i,j,k,l))
edge_val_constraints = [Or(edge_val==0,edge_val==1) for edge_val in
    edge_var.values()]
\end{lstlisting}
\doublespace

In our construction of the directed graph, there can be at most one outgoing edge from every square, meaning that the sum of all \t{edge\_val} for a cell will always be less than or equal to 1.

\singlespace
\begin{lstlisting}[language=python, frame=single]
for i in range(N):
   for j in range(N):
      for (k,l) in getAdjacent1(i,j, N):
         if lessThan(i,j,k,l):
            sum_edges = edge_var[(i,j,k,l)] + edge_var[(k,l,i,j)]
                 edge_val_constraints.append(sum_edges <= 1)
\end{lstlisting}
\doublespace

We then add the constraint that if there is an edge between two cells,
those two cells are in the same regions; therefore, they should have
the same value. Finally, we define a cell size as being the sum of the
sizes of the adjacent cells + 1. In each region, we add the constraint
such that there is exactly one cell with a size equal to its value.

\subsubsection*{Creating the Initial Puzzle}
We have two ways of creating the initial full board. We generate a
full board with z3 using the declarative definition of a Fillomino
puzzle, but this code is slow and not random in z3. Our second option
is to create a board only using a Python program. The Python program works by
selecting a starting square, randomly selecting a sequence length from
a list of valid sequence lengths, and then setting a list of squares
that are sequence length long and that all have the value sequence
length. We continue with this selection process until the board is completely filled.


\subsubsection*{Emptying Squares}

We adapt our emptying algorithm for Fillomino.

\begin{enumerate}
\item {\bf Select a square to empty.} Choose a square from a region
  that has more than one cell that is not emptied. Unlike in Sudoku, we do not check probabilistically for squares in rows, columns, or three by three grids that have more un-emptied squares because the only constraint in Fillomino is that there must be at least one value in every region that is not emptied.
\item {\bf Temporarily set the selected square as emptied, check whether the temporary board is valid, and permanently remove a valid square.} These steps are the exact same as steps 2 - 4 of Sudoku emptying.
\item{\bf Repeat steps 1 and 2} until the target number of emptied squares is reached or the sum of the number of squares in \t{vals\_tried} and the number that are already emptied reaches $N^2$. 
\end{enumerate}

\subsubsection*{Transformations}
Because of the region constraints, there are less
transformations that can be applied to Fillomino than Sudoku, but a
number still exist. These are (1) Rotation, (2) Vertical reflection, and (3) Horizontal reflection. These transformations be applied to an emptied board to get 7 new Fillomino boards with different orientations form the original one.

\subsubsection*{Defining Complexity}
Using machine learning, we determine the difficulty of an unsolved
Fillomino puzzle based on its 4-component characterizing vector. The components are (1) number of cells; (2) number of empty squares; (3) number of regions; (4) optimality, whether the puzzle can be further emptied. The function generated by the SVM categorizes a Fillomino puzzles into four difficulties similar to those of Sudoku puzzles.

